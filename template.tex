%%%%%%%%%%%%%%%%%%%%%%%%%%%%%%%%%%%%%%%%%
% Twenty Seconds Resume/CV
% LaTeX Template
% Version 1.0 (14/7/16)
%
% Original author:
% Carmine Spagnuolo (cspagnuolo@unisa.it) with major modifications by
% Vel (vel@LaTeXTemplates.com) and Harsh (harsh.gadgil@gmail.com)
%
% License:
% The MIT License (see included LICENSE file)
%
%%%%%%%%%%%%%%%%%%%%%%%%%%%%%%%%%%%%%%%%%

%----------------------------------------------------------------------------------------
%	PACKAGES AND OTHER DOCUMENT CONFIGURATIONS
%----------------------------------------------------------------------------------------

\documentclass[letterpaper]{twentysecondcv} % a4paper for A4

%----------------------------------------------------------------------------------------
%	 PERSONAL INFORMATION
%----------------------------------------------------------------------------------------
% If you don't need one or more of the below, just remove the content leaving the command, e.g. \cvnumberphone{}

\cvname{Antonio Riccio}
\cvjobtitle{ Embedded Engineer } % Job title/career

\cvnumberphone{(+44) 7724 799572}
\cvmail{antonio.riccio.27@gmail.com}
\cvlinkedin{/in/antonioriccio}
\cvgithub{artic92}

%----------------------------------------------------------------------------------------
%	INTERESTS SECTION
%----------------------------------------------------------------------------------------

% Command for printing skill overview bubbles
\interests{
~
	\smartdiagram[bubble diagram]{
        \textbf{Embedded}\\\textbf{Systems},
        \textbf{~~~~~~Linux~~~~~~}\\\textbf{~~~~~Kernel~~~~~},
        \textbf{~~~~FPGA~~~~}\\\textbf{~~~SoC~~~},
        \textbf{~~~~~~Safety~~~~~~}\\\textbf{~~~Systems~~~},
        \textbf{~~~~RTOS~~~~},
        %\textbf{Hardware}\\\textbf{Security},
        \textbf{Trust}\\\textbf{Computing},
        %\textbf{CPU}\\\textbf{Architectures},
        \textbf{Internet of}\\\textbf{~~Things~~}
        %\textbf{~~Device~~}\\\textbf{~~Drivers~~}
    }
}

%----------------------------------------------------------------------------------------
%	SKILLS SECTION
%----------------------------------------------------------------------------------------

\skills{
    % Soft Skills
    %{Adaptable $\textbullet$ Open-minded $\textbullet$ Rigorous / 6},
    {Positive $\textbullet$ Team-player $\textbullet$ Determined / 6},
    {Curious $\textbullet$ Motivated $\textbullet$ Passionate / 6},
    % Documentation writing
    %{Markdown $\textbullet$ Doxygen $\textbullet$ \large \LaTeX / 6},
    % Scripting languages
    %{Bash $\textbullet$ Python/ 6},
    % OSs & RTOSs
    {Linux $\textbullet$ FreeRTOS $\textbullet$ RTAI $\textbullet$ eCos / 6},
    % Safety-related tools
    {FMEA $\textbullet$ RBD $\textbullet$ Fault Trees $\textbullet$ FFDA / 6},
    % HW Platforms
    {CPU $\textbullet$ MCU $\textbullet$ SoC $\textbullet$ FPGA $\textbullet$ ASIC/ 6},
    % Vendors
    %{ARM $\textbullet$ Xilinx $\textbullet$ ST Microelectronics/ 6},
    % Serial buses
    {I2C $\textbullet$ SPI $\textbullet$ USART $\textbullet$ AXI $\textbullet$ USB/ 6},
    % Programming languages
    {C $\textbullet$ C++ $\textbullet$ Assembler $\textbullet$ Java/ 6},
    % HDLs
    {VHDL $\textbullet$ SystemVerilog/ 6}}

%----------------------------------------------------------------------------------------
%	EDUCATION SECTION (IN SIDEBAR)
%----------------------------------------------------------------------------------------

\education{
    \textbf{M.Sc. Computer Engineering} | \\
    \textit{Embedded \& Industrial Systems} \\
    (110/110 \textit{cum laude}) \\
    University Federico II \\
    Naples, Italy | 2015 - 2019

    \vspace{1mm}

    \textbf{B.Sc. Computer Engineering} \\
    (107/110) \\
    University Federico II \\
    Naples, Italy | 2011 - 2015
}

%----------------------------------------------------------------------------------------
%	VOLUNTEERING SECTION
%----------------------------------------------------------------------------------------

\volunteering{}

%----------------------------------------------------------------------------------------

\begin{document}

\makeprofile % Print the sidebar

%----------------------------------------------------------------------------------------
%	 EXPERIENCE
%----------------------------------------------------------------------------------------

\section{Experience}

\begin{twenty} % Environment for a list with descriptions
    \twentyitem
    	{May 2019 -}
		{now}
        {Junior Embedded Software Engineer}
        {\href{https://www.airspan.com/}{Airspan Networks, London}}
        {UK embedded department. Manager: \textbf{Ashvtovsh Goel}}
        {
            \begin{itemize}
                \item Deisgn and develop embedded software for \textit{Airunity} line of products\\
                \textit{Tools}: \textbf{Jira}, \textbf{Gerrit}, \textbf{Agile}, \textbf{i.MX}, \textbf{iptables}, \textbf{nftables}, \textbf{ebtables} 
            \end{itemize}
        }\\
\twentyitem
    	{Jan 2018 -}
		{Aug 2018}
        {Software Engineer Intern}
        {\href{https://www.montimage.com/}{Montimage, Paris}}
        {}
        {
            \begin{itemize}
                \item Realization of an Intrusion Detection System for the IoT\\
                Helped the company to start the IoT business\\
                Project: \href{http://www.anastacia-h2020.eu/}{\textbf{EU Horizon 2020 ANASTACIA}}\\
                \textit{Tools}: \textbf{Apache Kafka}, \textbf{Apache Storm}, \textbf{Cooja}, \textbf{Contiki}, \textbf{Git}
                % \item Presented the work at talks and project meetings
                % \item Produced project deliverables
                % \item Collaborated with other partners for integration
            \end{itemize}
        }\\
    \twentyitem
   		{Sep 2017 -}
		{Jan 2018}
        {Graduate Teaching Assistant}
        {\href{http://www.scuolapsb.unina.it/}{DIETI, University of Naples}}
        {}
        {
            \begin{itemize}
                %\item Programming Fundamentals and Mathematical Analysis I courses
                \item Delivered lectures to students from undergraduate courses
                \item Preparing lab materials including assignments and the final exam
            \end{itemize}
        }
	%\twentyitem{<dates>}{<title>}{<location>}{<description>}
\end{twenty}

%----------------------------------------------------------------------------------------
%	 PROJECTS
%----------------------------------------------------------------------------------------

\section{Projects}

\begin{twenty}
%	\twentyitem
%    	{Oct 2017 -}
%		{Dec 2017}
%        {Nu+: an open-source GPU-like processor core (SystemVerilog)}
%        {}
%        {}
%        {
%            \begin{itemize}
%                \item Deep understanding of Cache Coherence Subsystem functioning
%                \item Aligning existing documentation to the newest advancements
%                \item Producing documentation sections on the newest subsystems
%		    \end{itemize}
%		}\\
	\twentyitem
    	{Mar 2017 -}
		{Jul 2017}
        {GNSS Synchronisation \& Bistatic Passive Radar (VHDL,C)}
        {\href{https://github.com/artic92/sistemi-embedded-task2}{Github}}
        {}
        {
            \begin{itemize}
                \item Realization of IP cores for the subsystem in charge of synchronizing with a reference GNSS satellite. Working in a team of 3. \\
                \textit{Tools}: \textbf{Xilinx Vivado}, \textbf{Modelsim}, \textbf{Zynq-7000 ARM/FPGA SoC}, \textbf{Linux}
                %\item Produced internal and external documentation
                \item Collaborated with other teams for integration
                %\item Presented the results to the stakeholders
		    \end{itemize}
        }\\
	\twentyitem
    	{Sep 2015 -}
		{Dec 2015}
        {Arithmetic Logic Unit (ALU) Implementation (VHDL)}
        {\href{https://github.com/artic92/alu_xilinx}{Github}}
        {}
        {
            \begin{itemize}
                \item In a team of 3, realisation of a custom ALU\\
                \textit{Tools}: \textbf{Xilinx ISE}, \textbf{Modelsim}, \textbf{Timing Analyser}, \textbf{Spartan 3E FPGA}
                \item Generalised desing approach
                \item Evaluation in terms of performance, area, and delay
                \item Project presentation
		    \end{itemize}
        }\\
    \twentyitem
    	{Sep 2015 -}
		{Dec 2015}
        {GPIO Custom Implementation (C,VHDL)}
        {\href{https://github.com/artic92/gpio-zynq-7000}{Github}}
        {}
        {
            \begin{itemize}
                \item Realization of a custom GPIO\\
                \textit{Tools}: \textbf{Xilinx Vivado}, \textbf{Modelsim}, \textbf{Zynq ARM SoC}, \textbf{ARM AMBA/AXI}
                \item Description of the hardware in VHDL
                \item Implemented a custom BSP for the device
                \item Implementating of device drivers in MMAP, UIO, and Kernel Module
		    \end{itemize}
        }\\
    \twentyitem
    	{Mar 2015 -}
		{Jun 2015}
        {Triple Modular Redundancy (TMR) Scheme (C,RTOS)}
        {\href{https://github.com/artic92/tmr_rtai}{Github}}
        {}
        {
        {
            \begin{itemize}
                \item In a team of 2, realisation of a software TMR scheme. \\\textit{Tools}: \textbf{RTAI RTOS}.
                \item Schedulability assessment
                \item Faults injection testing
                \item Produced internal and external documentation
		    \end{itemize}
		}
        }
\end{twenty}

%----------------------------------------------------------------------------------------
%	 RESEARCH
%----------------------------------------------------------------------------------------
\section{Research}
\begin{twenty}
	\twentyitem
    	{Sep 2018 -}
		{Jan 2019}
        {M.Sc. Candidate}
        {\href{http://www.scuolapsb.unina.it/}{University of Naples}}
        {}
        {
       	\textbf{Paper}: A Security Monitoring System for Internet of Things.  \textit{Internet of Things: Engineering Cyber Physical Human Systems}, 2019 \\
       	\textbf{Thesis}: Design of an Intrusion Detection System for IoT Security
        {
            \begin{itemize}
                \item Proposed an Intrusion Detection System for 6LoWPAN-based IoT networks, in collaboration with the Montimage company\\
                \textit{Tools}: \textbf{MMT}, \textbf{Contiki}, \textbf{Cooja}, \textbf{C}, \textbf{Python}, \textbf{Bash}
                \item Analysis of the open challenges, reference architectures, vulnerabilities in the context of IoT security
                \item Deep understanding of the IEEE 802.15.4 technology and the 6LoWPAN protocol stack
		    \end{itemize}}
        }
\end{twenty}
%
%\section{Publications}
%V. Casola, A. De Benedictis, A. Riccio, D. Rivera, W. Mallouli, and E. Montes De Oca, “A security monitoring system for Internet of Things”. \textit{Internet of Things: Engineering Cyber Physical Human Systems}, 2019. \textbf{Under review}. %\vspace{2mm}
%
\end{document}
