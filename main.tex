%%%%%%%%%%%%%%%%%%%%%%%%%%%%%%%%%%%%%%%%%
% Twenty Seconds Resume/CV
% LaTeX Template
% Version 1.0 (14/7/16)
%
% Original author:
% Carmine Spagnuolo (cspagnuolo@unisa.it) with major modifications by
% Vel (vel@LaTeXTemplates.com) and Harsh (harsh.gadgil@gmail.com)
%
% License:
% The MIT License (see included LICENSE file)
%
%%%%%%%%%%%%%%%%%%%%%%%%%%%%%%%%%%%%%%%%%

%----------------------------------------------------------------------------------------
%	PACKAGES AND OTHER DOCUMENT CONFIGURATIONS
%----------------------------------------------------------------------------------------

\documentclass[letterpaper]{twentysecondcv} % a4paper for A4

%----------------------------------------------------------------------------------------
%	 PERSONAL INFORMATION
%----------------------------------------------------------------------------------------

\cvname{Antonio Riccio}
\cvjobtitle{\textbf{Embedded Systems} \\ \textbf{Engineer}}

% \cvnumberphone{(+44) 7724 799572}
\cvnumberphone{(+49) 151 56893596}
\cvmail{ariccio.careers@outlook.com}
\cvlinkedin{antonioriccio}
\linkedinnameshown{Antonio Riccio}
\cvgithub{artic92}
\cvaddress{Munich, Germany}
\cvtwitter{AntonioRiccio27}
%\aboutme{\summaryfirst}
\aboutme{\summarythird}
\notes{\textit{Available to relocate}}

\newcommand{\summaryfirst}[1]{
Young Embedded Engineer striving to commence a career in the space industry. Amazed about everything crosses space: from satellites and probes to rockets and spaceships. My passion is to discover each detail concerning the reliability and safety of computer hardware \& software. Enthusiast about Linux Kernel and hard believer in the power of bash to solve each problem.
}

\newcommand{\summarysecond}[1]{
Passionate, motivated, curious and ambitious are my most identifying traits. I am very fond of astronomy and physics. I love reading books regarding relativistic \& quantum mechanics, astrophysics, and psychology. I enjoy doing sport as well. I practice jogging but I love to swim and play football also.
}

\newcommand{\summarythird}[1]{
As a natural-born self-starter, I am dedicated, diligent, reliable, and a passionate team player looking to put 100\% of myself. I
am a follower of science and technology. Being a chess player, one of the critical skills that I value the most is 'Patience'. Thus, I am committed to going the extra mile to understand how things work under the hood.
}

\newcommand{\loremipsum}[1]{Lorem ipsum dolor sit amet, consectetur adipiscing elit, sed do eiusmod tempor incididunt ut labore et dolore magna aliqua. Ut enim ad minim veniam, quis nostrud exercitation ullamco laboris nisi ut aliquip ex ea commodo consequat. Duis aute irure dolor in reprehenderit in voluptate velit esse cillum dolore eu fugiat nulla pariatur. Excepteur sint occaecat cupidatat non proident, sunt in culpa qui officia deserunt mollit anim id est laborum. Vulputate mi sit amet mauris commodo. Volutpat ac tincidunt.}

%----------------------------------------------------------------------------------------
%	INTERESTS
%----------------------------------------------------------------------------------------

% Command for printing skill overview bubbles
\interests{
	\smartdiagram[bubble diagram]{
        \textbf{Embedded}\\\textbf{Systems},
        \textbf{~~~~~~Linux~~~~~~}\\\textbf{~~~~~Kernel~~~~~},
        \textbf{~~~~FPGA~~~~}\\\textbf{~~~SoC~~~},
        \textbf{~~~~~~Safety~~~~~~}\\\textbf{~~~Systems~~~},
        \textbf{~~~~RTOS~~~~},
        %\textbf{Hardware}\\\textbf{Security},
        % \textbf{Trust}\\\textbf{Computing},
        \textbf{~~~~Avionics~~~~},
        %\textbf{CPU}\\\textbf{Architectures},
        \textbf{Internet of}\\\textbf{~~Things~~}
        %\textbf{~~Device~~}\\\textbf{~~Drivers~~}
    }
}

%----------------------------------------------------------------------------------------
%	SKILLS
%----------------------------------------------------------------------------------------

%\skills{
    % Soft Skills
    %{Adaptable $\textbullet$ Open-minded $\textbullet$ Rigorous / 6},
    %{Positive $\textbullet$ Team-player $\textbullet$ Determined / 6},
    %{Curious $\textbullet$ Motivated $\textbullet$ Passionate / 6},
    % Documentation writing
    %{Markdown $\textbullet$ Doxygen $\textbullet$ \large \LaTeX / 6},
    % Scripting languages
    %{Bash $\textbullet$ Python/ 6},
    % OSs & RTOSs
   % {Linux $\textbullet$ FreeRTOS $\textbullet$ RTAI $\textbullet$ eCos / 6},
    % Safety-related tools
  %  {FMEA $\textbullet$ RBD $\textbullet$ Fault Trees $\textbullet$ FFDA / 6},
    % HW Platforms
 %   {CPU $\textbullet$ MCU $\textbullet$ SoC $\textbullet$ FPGA $\textbullet$ ASIC/ 6},
    % Vendors
    %{ARM $\textbullet$ Xilinx $\textbullet$ ST Microelectronics/ 6},
    % Serial buses
%    {I2C $\textbullet$ SPI $\textbullet$ USART $\textbullet$ AXI $\textbullet$ USB/ 6},
    % Programming languages
%    {C $\textbullet$ C++ $\textbullet$ Assembler $\textbullet$ Java/ 6},
%    % HDLs
%    {VHDL $\textbullet$ SystemVerilog}
%}

%%%%%%%%%%%%%%%%%%%%%%%%%%%%%%%%%%%%%%%%%%%%%%%%%%%%%%%%%%%%%%
%%%%%%Skill bar section, each skill must have a value between 0 an 6 (float)%%%%%%%
%%%%%%%%%%%%%%%%%%%%%%%%%%%%%%%%%%%%%%%%%%%%%%%%%%%%%%%%%%%%%%
\skills{
    {C $\textbullet$ Bash $\textbullet$ VHDL $\textbullet$ Linux $\textbullet$ C++/4.5},
    {Java $\textbullet$ MCU $\textbullet$ SoC $\textbullet$ FPGA/4},
    {I2C $\textbullet$ SPI $\textbullet$ USART $\textbullet$ AXI $\textbullet$ USB/3.8},
    {Markdown $\textbullet$ Doxygen $\textbullet$ \LaTeX/3.5},
    {CPU $\textbullet$ Assembler $\textbullet$ RISC/3.3},
    {SystemVerilog $\textbullet$ freeRTOS $\textbullet$ RTAI/3}}

\softskills{
    {Curious $\textbullet$ Motivated $\textbullet$ Rigorous/4.5},
    {Passionate $\textbullet$ Positive $\textbullet$ Team Player/4.5},
    {Adaptable $\textbullet$ Open-minded $\textbullet$ Patient/4}}

%----------------------------------------------------------------------------------------
%	LANGUAGES
%----------------------------------------------------------------------------------------

\languages{
    \begin{itemize}
        \item \textbf{Italian}\hspace{3.3mm}| mother tongue
        \item \textbf{English}\hspace{2mm}| C1
        \item \textbf{German}\hspace{2.8mm}| A2
        \item \textbf{French}\hspace{2.8mm}| A1
    \end{itemize}
}

%----------------------------------------------------------------------------------------
%	CERTIFICATIONS
%----------------------------------------------------------------------------------------

\certifications{
    % \href{https://www.cambridgeenglish.org/exams-and-tests/advanced/}{\textbf{Certificate in Advanced English (CAE)}} |
    % \textit{Cambridge Assessment English} | 2019
    %
    % \vspace{0.5mm}
    %
    C1: \href{https://courses.edx.org/certificates/cceafff55a9b42c28bd1c0056e1d574c}{\textbf{Engineering the Space Shuttle}} | \textit{edX}

    \vspace{0.5mm}

    C2: \href{https://courses.edx.org/certificates/1952c815bf1a48b09b4bdc7e5ec62587}{\textbf{Space Mission Design and Operations}} | \textit{edX}

    \vspace{0.5mm}

    C3: \href{https://courses.edx.org/certificates/fe4896ee42de4e28a9951475845face8}{\textbf{Aerospace Engineering: Astronautics and Human Spaceflight}} | \textit{edX}
    %
    %\vspace{0.5mm}
    %
    %\href{https://www.edx.org/course/the-conquest-of-space-space-exploration-and-rocket}{\textbf{The Conquest of Space: Space Exploration and Rocket Science}} | \textit{edX} | 2020
}

%----------------------------------------------------------------------------------------
%	EDUCATION
%----------------------------------------------------------------------------------------

\education{
    \textbf{M.Sc. Computer Engineering} | \\
    \textit{Embedded \& Industrial Systems} \\
    (110/110 \textit{cum laude}) \\
    University Federico II \\
    Naples, Italy | 2015 - 2019

    \vspace{1mm}

    \textbf{B.Sc. Computer Engineering} \\
    (107/110) \\
    University Federico II \\
    Naples, Italy | 2011 - 2015
}

%----------------------------------------------------------------------------------------
%	VOLUNTEERING
%----------------------------------------------------------------------------------------

\volunteering{
    \href{http://www.erasmusnapoli.it/}{\textbf{Erasmus Student Association (ESA)}} |
    Supporting incoming Erasmus students
    %\textit{Associate} | Naples, 2019

    \vspace{1mm}

    \href{https://www.goodgym.org/about}{\textbf{GoodGym Slough}} |
    Charity running \\ group supporting local community
    %\textit{Associate} | Slough, 2019
}

%----------------------------------------------------------------------------------------

\begin{document}
\makeprofile % Prints the sidebar

%
%----------------------------------------------------------------------------------------
%	 EXPERIENCE
%----------------------------------------------------------------------------------------
%
\section{Experience}

\begin{twenty} % Environment for a list with descriptions
    % \twentyitem
    %     {May 2022 -}
    %     {now}
    %     {TITLE}
    %     {\href{URL}{COMPANY, CITY, \textbf{COUNTRY}}}
    %     {}
    %     {
    %         BRIEF DESCRIPTION
    %         \vspace{1 mm}
    %         \begin{itemize}
    %             \item item 1
    %             \item item 2
    %             \item item 3
    %             \item item 4
    %             \item item 5
    %         \end{itemize}

    %         \vspace{1 mm}
    %         \textit{Keywords}: KEYWORDS
    %     }
    \twentyitem
        {May 2023 -}
        {Oct 2023}
        {Flight Software Engineer}
        {\href{https://www.exploration.space/}{The Exploration Company, Munich, \textbf{Germany}}}
        {}
        {
            Flight control software for the \textit{Mission Possible} spacecraft.
            \vspace{1 mm}
            \begin{itemize}
                \item Developed applications  using NASA's cFS framework.
                \item Ensured code compliance with MISRA and JPL guidelines.
                \item Enabled successful testing of applications by leveraging serial emulation tools and Python scripts, proving the software's functionality even in the absence of physical sensors.
            \end{itemize}

            \vspace{1 mm}
            \textit{Keywords}: \textbf{C}, \textbf{Python}, \textbf{Linux}, \textbf{Docker}, \textbf{Hardware-in-the-loop}, \textbf{Agile}.
        }\\
    \twentyitem
        {May 2022 -}
        {May 2023}
        {FCS Embedded Software Engineer}
        {\href{https://www.airbus.com/en/who-we-are}{Airbus Defence \& Space, Manching}}
        {}
        {
            My contribution as part of the Software Team is to to participate in all the phases
            of the development of the FCS.
            \vspace{1 mm}
            \begin{itemize}
                \item analysing requirements specification
                \item design test-cases adopting the MC/DC coverage approach
                \item execute tests with VectorCast
                \item tuning test-cases to improve coverage
            \end{itemize}

            \vspace{1 mm}
            \textit{Keywords}: \textbf{avionics}, \textbf{unit testing}, \textbf{ADA}, \textbf{DO-178B}, \textbf{VectorCast}, \textbf{MC/DC}
        }\\
    \twentyitem
    	{May 2020 -}
		{Oct 2021}
        {Avionics Digital Design Engineer}
        {\href{http://evoleotech.com/company/}{EVOLEO Technologies, Munich, \textbf{Germany}}}
        {}
        {
            \textbf{\href{https://sci.esa.int/web/plato}{PLATO - PLAnetary Transits and Oscillations of stars}}\\
            Realisation of the
            \href{https://platomission.com/2018/05/15/ancillary-electrical-units-aeu-2/}{\textbf{Ancillary
            Electrical Units (AEU)}} for the Service Module. Customer is the
            \textbf{European Space Agency (ESA)}.
            \vspace{1 mm}
            \begin{itemize}
                \item FPGA design \& development compliant with \textbf{ECSS-Q-ST-60-02C}
                % \item VHDL modelling according to \textbf{ESA ASIC/001} guidelines
                % \item supported automated FPGA functional verification
                \item integration of \textbf{Microsemi} and \textbf{ESA IP Cores}
                \item working with \textbf{SpaceWire RMAP} as per \textbf{ECSS-E-ST-50-52C}
                \item working with \textbf{AMBA 3 AHB-Lite} specification
                \item mentored junior engineers
                \item managed code versioning and releasing
                \item production of project deliverables
                \item day-to-day activities planning
                \item digital design for \textbf{Microsemi RTAX-S/SL} radiation-tolerant \textbf{FPGA}s
                \item digital design for \textbf{Microsemi proASIC} industrial \textbf{FPGA}s
            \end{itemize}

            \vspace{1 mm}
            \textbf{CHICS - COTS-based Highly Integrated Computer Systems}\\
            CHICS aims to develop an On-Board Computer, based on \textbf{COTS} and
            \textbf{MPSoC}, that is easily adaptable to the needs of
            different missions. Customer is \textbf{Airbus Defence \& Space GmbH}.
            \vspace{1 mm}
            \begin{itemize}
                \item architectural design compliant with the \textbf{SAVOIR} specification
                \item working with the \textbf{CCSDS 133.0-B-2} space packet protocol
                % \item working with \textbf{PUS} services
                \item digital design for \textbf{Xilinx Zynq UltraScale+ MPSoC}
                \item digital design for \textbf{Microsemi PolarFire FPGA}
            \end{itemize}

            \vspace{1 mm}
            \textit{Keywords}: \textbf{ECSS}, \textbf{VHDL}, \textbf{SpaceWire}, \textbf{Xilinx}, \textbf{Microsemi}, \textbf{FPGA}, \textbf{MPSoC}, \textbf{Vivado}, \textbf{SAVOIR}, \textbf{avionics}, \textbf{proASIC}, \textbf{RTAX}, \textbf{Zynq UltraScale+}.
        }\\
    \twentyitem
    	{May 2020 -}
		{May 2019}
        {Junior Embedded Software Engineer}
        {\href{https://www.airspan.com/}{Airspan Communications, London, \textbf{UK}}}
        {}
        {
            Integration of the \textbf{Linux kernel} into the mainline products. Development of application software for the \textit{Airunity} and the \textit{Air-To-Ground (ATG)} family of products.
            \vspace{1 mm}
            \begin{itemize}
                % \item design and development of user-space software for the \textbf{Linux kernel}
                % \item development of user-space software for the \textbf{Linux kernel}
                % \item development of applications using the Linux \textbf{Input Subsystem}
                % \item development of user-space applications using the \textbf{I2C} Subsystem
                % \item design and development of \textbf{kernel-space} software
                % \item working on \textit{NXP}, \textit{Qualcomm} and \textit{Xilinx} SoCs
                % \item working on \textbf{Xilinx} SoCs
                % \item upgrading custom modules to match the latest kernel API changes
                % \item development of \textbf{device drivers} to integrate new devices
                \item \textbf{peer-reviewing} code before merging into the codebase
                % \item setup of the target platform for subsequent system testing
                \item \textbf{unit and system testing} on the target platform
                % \item extensive system testing onto different models
                % \item design and development of application software in \textbf{object-based C}
                % \item design and development of automatic \textbf{unit-tests} in C++
                % \item debugging software for investigation/studying purposes
                % \item automated and manual \textbf{log analysis} for debugging purposes
                \item solution design through \textbf{UML} documentation
                % \item setup of the target platform for subsequent system testing
                \item testing and log collection from the target platform
                % \item software testing and deployment on the real devices
                \item development of distributed application software using \textbf{Unix sockets}
                \item debugging adopting network analyzers (Wireshark, tcpdump, iperf)
                % \item participation in regular \textbf{project meetings}
                \item showing advancements through presentations and demos
                \item realisation of a \textbf{containerised} simulation environment with \textbf{Docker}
                % \item \textbf{peer-reviewing} code before merging into the codebase
            \end{itemize}

            \vspace{1 mm}
            \textit{Keywords}: \textbf{SoC}, \textbf{FPGA}, \textbf{i.MX}, \textbf{Xilinx}, \textbf{Zynq-7000}, \textbf{C}, \textbf{Linux}, \textbf{Docker}, \textbf{C}, \textbf{C++}, \textbf{Docker}, \textbf{UML}, \textbf{Modelio}, \textbf{Unix Domain Socket}, \textbf{Git}.
            % UNUSED
            % \textbf{ARM}, \textbf{Bash}, \textbf{Coding Conventions}, \textbf{GDB}, \textbf{Simulation}, \textbf{NXP}
        }\\
    % \twentyitem
    % 	{Jan 2018 -}
	% 	{Aug 2018}
    %     {Software Engineer Intern}
    %     {\href{https://www.montimage.com/}{Montimage, Paris, \textbf{France}}}
    %     {}
    %     {
    %         Realization of an \textbf{Intrusion Detection System (IDS)} for the \textbf{IoT}. Activity framed in the context of the \href{http://www.anastacia-h2020.eu/}{\textbf{ANASTACIA} EU H2020} project.
    %         \vspace{1 mm}
    %         \begin{itemize}
    %             % \item design and development of application software in procedural C
    %             % \item integration and reuse of legacy solutions from previous projects
    %             \item debugging software using \textbf{GDB}
    %             \item preliminary integration testing within \textbf{Cooja} network simulator
    %             % \item benign and malicious \textbf{traffic generation} for testing phases
    %             \item analysis of network traffic with \textbf{Wireshark} and \textbf{Tcpdump}
    %             % \item definition of a \textbf{simulated target environment} with virtual machines
    %             % \item integration testing on the simulated environment
    %             % \item deployment and integration testing on the real target environment
    %             \item integration and system testing with ANASTACIA partners
    %             % \item performed system testing collaborating with ANASTACIA partners
    %             % \item log gathering from the target for subsequent study
    %             \item participation in \textbf{conference calls} and \textbf{projects meetings}
    %             \item \textbf{contribution} to ANASTACIA project \textbf{deliverables}
    %         \end{itemize}

    %         \vspace{1 mm}
    %         \textit{Keywords}: \textbf{C}, \textbf{GDB}, \textbf{Cooja}, \textbf{IoT}, \textbf{IDS}, \textbf{Deep Packet Inspection}, \textbf{Signature-based Detection}, \textbf{Project Deliverable}, \textbf{Cybersecurity}.
    %         % UNUSED
    %         % \textbf{Python}, \textbf{Git}, \textbf{Markdown}, \textbf{Holistic Security}, \textbf{Simulation}, \textbf{Virtual Machine}, \textbf{ICMP Flood}, \textbf{SQL Injection}, \textbf{Wireshark}
    %     }
%     \twentyitem
%   		{Sep 2017 -}
% 		{Jan 2018}
%         {Graduate Teaching Assistant}
%         {\href{http://www.scuolapsb.unina.it/}{DIETI, University of Naples, \textbf{Italy}}}
%         {}
%         {
%             Delivered \textbf{lectures} to students from undergraduate courses
%             \begin{itemize}
%                 \item Presenting concepts of \textbf{object-oriented programming} and \textbf{software testing}
%                 \item Presenting concepts of \textbf{mathematical analysis}
%                 \item Preparing lab materials including \textbf{assignments}, and the \textbf{final exam}
%             \end{itemize}
%         }
\end{twenty}

\newpage
\makeextrainfo % Prints the sidebar

%----------------------------------------------------------------------------------------
%	 RESEARCH
%----------------------------------------------------------------------------------------
\section{Research}
\begin{twenty}
	\twentyitem
    	{Jun 2021 -}
		{Aug 2021}
        {Professional Degree in Space Studies}
        {\href{https://www.isunet.edu/}{ISU, Strasbourg, \textbf{France}}}
        {\href{https://www.isunet.edu/ssp/}{\textit{Space Studies Program 2021}}}
        {
            \textbf{Specialisation}: \textit{Management and Business (MGB)}. In a team of 7, presented \textit{Astrolaunch}. Produced business plan, pitch deck and teaser. Presented the idea in front of jury of investors and experts.\\
            \textbf{Team Project}: \textit{Moon on-orbit Nexus providing orbital rendezvous and transportation}, MOONPORT. Innovative space transportation system to the cis-lunar space. Leading the Science team.

            \vspace{1 mm}
            \textit{Keywords}: \textbf{Start-up}, \textbf{Public Speaking}, \textbf{Team-leading}, \textbf{Moon Exploration}, \textbf{On-orbit Servicing}, \textbf{New Space}, \textbf{Space Tug}.
            % \textbf{Propulsion}, \textbf{Space Industrialization}, \textbf{Space Robotics}, \textbf{Space Vehicles Rendezvous and Docking}, \textbf{Space Vehicles Maintenance and Repair}.
        }\\
	\twentyitem
    	{Sep 2018 -}
		{Jan 2019}
        {M.Sc. Candidate}
        {\href{http://www.scuolapsb.unina.it/}{University of Naples, Federico II, \textbf{Italy}}}
        {}
        {
           Proposed an \textbf{Intrusion Detection System} (IDS) for 6LoWPAN-based IoT networks in collaboration with the Montimage company and in the context of the ANASTACIA EU project.\\
            \textbf{Thesis}: \textit{Design of an Intrusion Detection System for IoT Security}
            % {
            %     \begin{itemize}
            %         \item analysed the consequences of a not universally acknowledged definition of the IoT
            %         \item surveyed the most diffused \textbf{IoT Reference Architectures}, in particular the \textbf{Cisco IoT Reference Model}
            %         \item focused on the Connectivity Level of the IoT Reference Model
            %         \item analysed the security model of the IEEE 802.15.4 protocol
            %         \item analysed the security offered at each level of the 6LoWPAN stack
            %         \item surveyed \textbf{vulnerabilities} and \textbf{threats} according to the IAS Octave classification per each level of the 6LoWPAN stack
            %         \item compared the \textit{state-of-the-art} in terms of IDS for the IoT
            %         \item \textbf{experimental tests} on the ANASTACIA infrastructure
            %     \end{itemize}
            % }

            \vspace{1 mm}
            \textit{Keywords}: \textbf{Contiki}, \textbf{Cooja}, \textbf{IoT}, \textbf{IEEE 802.15.4}, \textbf{6LoWPAN}, \textbf{Research}, \textbf{Security}, \textbf{Cybersecurity}, \textbf{Paper}, \textbf{Publication}.
        }
\end{twenty}
%
%----------------------------------------------------------------------------------------
%	 PUBLICATIONS
%----------------------------------------------------------------------------------------
%
\section{Publications}
 F. Bergamasco, \href{https://scholar.google.com/citations?user=A3XqqTEAAAAJ&hl=it}{A. Riccio}, et. al.,
 \href{https://www.researchgate.net/publication/355856343_MOONPORT_A_cost-effective_transport_solution_for_cislunar_space}
 {\textbf{MOONPORT: A cost-effective transport solution for cislunar space}}.
 \href{https://iafastro.directory/iac/paper/id/63931/summary/}
 {\textit{72nd International Astronautical Congress}}, 2021.

 %V. Casola, A. De Benedictis, A. Riccio, D. Rivera, W. Mallouli, and E. Montes De Oca,
 V. Casola,  \href{https://scholar.google.com/citations?user=A3XqqTEAAAAJ&hl=it}{A. Riccio}, et al.,
 \href{https://www.researchgate.net/publication/334175322_A_security_monitoring_system_for_Internet_of_Things}
 {\textbf{A security monitoring system for Internet of Things}}.
 \textit{Internet of Things: Engineering Cyber Physical Human Systems}, 2019.

%
%----------------------------------------------------------------------------------------
%	 PROJECTS
%----------------------------------------------------------------------------------------
%
\section{Projects}

\textbf{GNSS Synchronisation \& Bistatic Passive Radar (VHDL, C)}: realization of IP cores for the subsystem in charge of synchronizing with a reference satellite \href{https://github.com/artic92/sistemi-embedded-task2}{(on \textit{GitHub})}.\\
\textit{Keywords}: \textbf{Vivado}, \textbf{Modelsim}, \textbf{Zynq ARM/FPGA SoC}, \textbf{Linux}.\\
\textbf{Arithmetic Logic Unit (ALU) Implementation (VHDL)}: realisation of a custom ALU using Xilinx toolchain \href{https://github.com/artic92/alu_xilinx}{(on \textit{GitHub})}.\\
\textit{Keywords}: \textbf{ISE}, \textbf{Modelsim}, \textbf{Timing Analyser}, \textbf{Spartan-3E FPGA}. \\
\textbf{GPIO Custom Implementation (C, VHDL)}: implementation of a custom GPIO using Xilinx Zynq-7000 ARM/FPGA SoC \href{https://github.com/artic92/gpio-zynq-7000}{(on \textit{GitHub})}.\\
\textit{Keywords}: \textbf{Vivado}, \textbf{Modelsim}, \textbf{Zynq ARM/FPGA SoC}, \textbf{AMBA/AXI}.\\
\textbf{Triple Modular Redundancy (TMR) Scheme (C, RTOS)}: realization of a TMR scheme in software using Linux RTAI OS  \href{https://github.com/artic92/tmr_rtai}{(on \textit{GitHub})}.\\
 \textit{Keywords}: \textbf{RTAI}, \textbf{RTOS}, \textbf{Linux kernel}, \textbf{Dependability}, \textbf{TMR}.\\
\textbf{Custom BSP for ST Microelectronics Microcontrollers (C)}: custom \textbf{Board Support Package} (BSP) for ST Microelectronics \textit{stm32fxx} family \href{https://github.com/artic92/stm32-bsp}{(on \textit{GitHub})}.\\
\textit{Keywords}: \textbf{Doxygen}, \textbf{ST CubeMX}, \textbf{Hardware Abstraction Layer} \\
\textbf{Nu+: an open-source GPU-like processor core (SystemVerilog)}: producing technical documentation for the Cache Coherence Subsystem \href{ http://www.naplespu.com/doc/index.php?title=Main_Page}{(\textit{project wiki})}.

%----------------------------------------------------------------------------------------
%	 COURSES
%----------------------------------------------------------------------------------------

\section{Courses}

\href{https://www.edx.org/course/engineering-the-space-shuttle}{\textbf{Engineering the Space Shuttle}}, Prof. \textit{Jeff Hoffman} |
\href{https://www.edx.org/course/the-conquest-of-space-space-exploration-and-rocket}{\textbf{Space Exploration and Rocket Science}}, Prof. \textit{Eduardo Ahedo Galilea} |
\textbf{Embedded Systems}, Prof. \href{https://www.docenti.unina.it/antonino.mazzeo}{\textit{Antonino Mazzeo}} |
\textbf{Digital Systems Design}, \href{https://www.researchgate.net/profile/Mario_Barbareschi}{\textit{Mario Barbareschi}} |
\textbf{Advanced Computer Architectures \& GPU Programming}, Prof. \href{https://www.docenti.unina.it/alessandro.cilardo}{\textit{Alessandro Cilardo}} |
\textbf{Computer Systems Dependability}, Prof. \href{https://www.docenti.unina.it/domenico.cotroneo}{\textit{Domenico Cotroneo}} |
\textbf{Software Engineering}, Prof. \href{https://www.docenti.unina.it/stefano.russo}{\textit{Stefano Russo}} |
\textbf{Numeric Calculus}, Prof. \href{https://www.docenti.unina.it/alessandra.dalessio}{\textit{Alessandra d'Alessio}} |
\textbf{Computer Architectures I}, Prof. \href{https://www.docenti.unina.it/antonio.pescape}{\textit{Antonio Pescapè}} |
\textbf{Computer Architectures II}, Prof. \href{https://www.docenti.unina.it/nicola.mazzocca}{\textit{Nicola Mazzocca}} |
\textbf{Algorithms and Data Structures}, Prof. \href{https://www.docenti.unina.it/stefano.avallone}{\textit{Stefano Avallone}} |
\textbf{Computer Networks}, Prof. \href{https://www.docenti.unina.it/giorgio.ventre}{\textit{Giorgio Ventre}} |
\textbf{Web Applications \& Streaming Technologies}, Prof. \href{https://www.docenti.unina.it/simonpietro.romano}{\textit{Simon Pietro Romano}} |
\textbf{Software Engineering II}, Prof. \href{https://www.docenti.unina.it/annarita.fasolino}{\textit{Anna Rita Fasolino}} |
\textbf{Programming Languages}, Prof. \href{https://www.docenti.unina.it/marcello.cinque}{\textit{Marcello Cinque}} |
\textbf{Relational Databases}, Prof. \href{https://www.docenti.unina.it/antonio.picariello}{\textit{Antonio Picariello}} |
\textbf{Distributed Systems}, Prof. \href{https://www.docenti.unina.it/stefano.russo}{\textit{Stefano Russo}} |
\textbf{Secure Systems Design}, Prof. \href{ttps://www.docenti.unina.it/mario.tanda}{\textit{Mario Tanda}}

\end{document}
