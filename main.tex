%!LW recipe=latexmk (xelatex)
%%%%%%%%%%%%%%%%%%%%%%%%%%%%%%%%%%%%%%%%%
% Twenty Seconds Resume/CV
% LaTeX Template
% Version 1.0 (14/7/16)
%
% Original author:
% Carmine Spagnuolo (cspagnuolo@unisa.it) with major modifications by
% Vel (vel@LaTeXTemplates.com) and Harsh (harsh.gadgil@gmail.com)
%
% License:
% The MIT License (see included LICENSE file)
%
%%%%%%%%%%%%%%%%%%%%%%%%%%%%%%%%%%%%%%%%%

%----------------------------------------------------------------------------------------
%	PACKAGES AND OTHER DOCUMENT CONFIGURATIONS
%----------------------------------------------------------------------------------------

\documentclass[letterpaper]{twentysecondcv} % a4paper for A4

%----------------------------------------------------------------------------------------
%	 PERSONAL INFORMATION
%----------------------------------------------------------------------------------------

\cvname{Antonio Riccio}
\cvjobtitle{\textbf{Embedded Systems} \\ \textbf{Engineer}}

\profilepic{img/headshot.jpg}
\cvnationality{}
\cvaddress{}
\cvnumberphone{(+49) 151 56893596}
\cvmail{careers@ariccio.me}
\cvsite{bit.ly/48mf8xO}
\cvsiteshown{ariccio.me}
\cvlinkedin{antonioriccio}
\linkedinnameshown{Antonio Riccio}
\cvgithub{}
\cvtwitter{}
\cvresearchgate{}
\aboutme{\summarytext}

\newcommand{\summarytext}[1]
{
    Blending coding expertise with a nomadic spirit, I, a software engineer, explore diverse landscapes, embracing finance and business realms with aspirations for leadership and a passion for new challenges.
}

%----------------------------------------------------------------------------------------
%	INTERESTS
%----------------------------------------------------------------------------------------

\interests{
	\smartdiagram[bubble diagram]{
        \textbf{Embedded}\\\textbf{Systems},
        \textbf{~~~~~~Linux~~~~~~}\\\textbf{~~~~~Kernel~~~~~},
        \textbf{~~~~FPGA~~~~}\\\textbf{~~~SoC~~~},
        \textbf{~~~~~~Safety~~~~~~}\\\textbf{~~~Systems~~~},
        \textbf{~~~~RTOS~~~~},
        \textbf{~~~~Avionics~~~~},
        \textbf{Internet of}\\\textbf{~~Things~~}
    }
}

%----------------------------------------------------------------------------------------
%	SKILLS
%----------------------------------------------------------------------------------------

% each skill must have a value between 0 an 6 (float)
\skills{
    {C $\textbullet$ Bash $\textbullet$ VHDL $\textbullet$ Linux $\textbullet$ C++/4.5},
    {Java $\textbullet$ MCU $\textbullet$ SoC $\textbullet$ FPGA/4},
    {I2C $\textbullet$ SPI $\textbullet$ USART $\textbullet$ AXI $\textbullet$ USB/3.8},
    {Markdown $\textbullet$ Doxygen $\textbullet$ \LaTeX/3.5},
    {CPU $\textbullet$ Assembler $\textbullet$ RISC/3.3},
    {SystemVerilog $\textbullet$ freeRTOS $\textbullet$ RTAI/3}}

\softskills{
    \smartdiagram[bubble diagram]{
        \textbf{~~~~Teamwork~~~~},
        \textbf{~~Problem~~}\\\textbf{~~Solving~~},
        \textbf{Time}\\\textbf{Management},
        \textbf{Leadership},
        \textbf{~~Learning~~}\\\textbf{~~Attitude~~},
        \textbf{Communication},
        \textbf{~~Critical~~}\\\textbf{~~Thinking~~}
    }
}

%----------------------------------------------------------------------------------------
%	LANGUAGES
%----------------------------------------------------------------------------------------

\languages{
    \begin{center}
        \worldflag[stretch=20, length=0mm, width=5mm]{IT} (m)
        \worldflag[stretch=20, length=0mm, width=5mm]{GB} (C2)
        \worldflag[stretch=20, length=0mm, width=5mm]{DE} (B1)
    \end{center}
}

%----------------------------------------------------------------------------------------
%	CERTIFICATIONS
%----------------------------------------------------------------------------------------

\certifications{
    \href{https://www.cambridgeenglish.org/exams-and-tests/advanced/}{\textbf{Certificate in Advanced English (CAE)}} |
    \textit{Cambridge Assessment English}

    \href{https://www.isunet.edu/event/executive-space-course-2024/}{\textbf{Executive Space Course}} |
    \textit{ISU}

    \href{https://coursera.org/share/021a99618cdf1997cabe7ab5b354c066}{\textbf{Financial Markets}} |
    \textit{Coursera}

    \href{https://courses.edx.org/certificates/1952c815bf1a48b09b4bdc7e5ec62587}{\textbf{Space Missions Design \& Ops}} | \textit{edX}

    \href{https://courses.edx.org/certificates/fe4896ee42de4e28a9951475845face8}{\textbf{Aerospace Eng.: Astronautics}} | \textit{edX}
}

%----------------------------------------------------------------------------------------
%	EDUCATION
%----------------------------------------------------------------------------------------

\education{
    \textbf{M.Sc. Computer Engineering} (\textit{laude})\\
    University Federico II, Italy | 2019

    \textbf{B.Sc. Computer Engineering} \\
    University Federico II, Italy | 2015
}

%----------------------------------------------------------------------------------------
%	VOLUNTEERING
%----------------------------------------------------------------------------------------

\volunteering{
    \href{https://spacebrewery.com/}{\textbf{SpaceBrewery}} |
    Organising networking events for space enthusiasts.

    \href{http://www.erasmusnapoli.it/}{\textbf{Erasmus Student Association (ESA)}} |
    Supporting incoming Erasmus students.

    \href{https://www.goodgym.org/about}{\textbf{GoodGym Slough}} |
    Charity running \\ group supporting local community.
}

%--------------------- DOCUMENT BEGINS --------------------------------------------------

\begin{document}
\makeprofile % Prints the sidebar

%
%----------------------------------------------------------------------------------------
%	 EXPERIENCE
%----------------------------------------------------------------------------------------
%
\section{Experience}

\begin{twenty} % Environment for a list with descriptions
    % \twentyitem
    %     {May 2022 -}
    %     {now}
    %     {TITLE}
    %     {\href{URL}{COMPANY, CITY, \textbf{COUNTRY}}}
    %     {}
    %     {
    %         BRIEF DESCRIPTION
    %         \vspace{1 mm}
    %         \begin{itemize}
    %             \item item 1
    %             \item item 2
    %             \item item 3
    %             \item item 4
    %             \item item 5
    %         \end{itemize}

    %         \vspace{1 mm}
    %         \textit{Keywords}: KEYWORDS
    %     }
    \twentyitem
        {Nov 2023 -}
        {May 2023}
        {Embedded Software Engineer}
        {\href{https://www.exploration.space/}{The Exploration Company, Munich, \textbf{Germany}}}
        {}
        {
            Flight control software for the \textit{Mission Possible} spacecraft.
            \vspace{1 mm}
            \begin{itemize}
                \item Developed applications  using NASA's cFS framework.
                \item Ensured code compliance with MISRA and JPL guidelines.
                \item Enabled successful testing of applications by leveraging serial emulation tools and Python scripts, proving the software's functionality even in the absence of physical sensors.
            \end{itemize}

            \vspace{1 mm}
            \textit{Keywords}: \textbf{C}, \textbf{Python}, \textbf{Linux}, \textbf{Docker}, \textbf{Hardware-in-the-loop}, \textbf{Agile}.
        }\\
    \twentyitem
        {Apr 2023 -}
        {May 2022}
        {Embedded Software Engineer}
        {\href{https://www.airbus.com/en/who-we-are}{Airbus Defence \& Space, Manching, \textbf{Germany}}}
        {}
        {
            Flight control software for the \textit{Eurofighter Typhoon} aircraft.
            \vspace{1 mm}
            \begin{itemize}
                \item Developed and updated test suites to ensure 100\% coverage and MC/DC criteria compliance.
                \item Conducted manual static analysis of the source code and assembler for potential safety hazards.
                \item Facilitated the successful integration of a critical component, enhancing aircraft performance and functionality.
            \end{itemize}

            \vspace{1 mm}
            \textit{Keywords}: \textbf{ADA}, \textbf{Assembler}, \textbf{DO-178}, \textbf{Unit Testing}, \textbf{Static Analysis}.
        }\\
    \twentyitem
        {Oct 2021 -}
    	{May 2020}
        {Digital Design Engineer}
        {\href{http://evoleotech.com/company/}{EVOLEO Technologies, Munich, \textbf{Germany}}}
        {}
        {
            Leading FPGA and SoC solutions for satellite applications.
            \vspace{1 mm}
            \begin{itemize}
                \item Spearheaded the digital design efforts for the ESA \href{https://sci.esa.int/web/plato}{PLATO} mission, streamlining development processes and implementing rigorous testing methodologies to ensure adherence to ESA standards.
                \item Successfully navigated the project through both the Preliminary Design Review and the Critical Design Review, addressing concerns and ensuring compliance with project requirements.
                \item Transitioned to a leadership role in a groundbreaking project focused on developing a satellite onboard computer using commercial components, overseeing digital design activities and coordinating with software teams for seamless integration.
            \end{itemize}

            \vspace{1 mm}
            \textit{Keywords}: \textbf{VHDL}, \textbf{C}, \textbf{SpaceWire}, \textbf{RMAP}, \textbf{CCSDS}, \textbf{ECSS}, \textbf{DSP}, \textbf{COTS}, \textbf{Xilinx US+}, \textbf{Leadership}, \textbf{Project Management}, \textbf{Mentorship}.
        }\\
    \twentyitem
    	{Apr 2020 -}
		{May 2019}
        {Embedded Software Engineer}
        {\href{https://www.airspan.com/}{Airspan, London, \textbf{UK}}}
        {}
        {
            Optimising testing processes and enhancing software scalability.
            \vspace{1 mm}
            \begin{itemize}
                \item Engineered a solution to automate testing of GPS modules, facilitating seamless integration into the company's testing framework.
                \item Engineered a solution for managing diverse touchscreens via I2C interface and Linux system calls, employing object-oriented C programming for enhanced software flexibility and scalability.
                \item Spearheaded the creation of a virtualised environment to simulate hardware platforms, ensuring uninterrupted application software testing even in the absence of physical hardware.
            \end{itemize}

            \vspace{1 mm}
            \textit{Keywords}: \textbf{C}, \textbf{C++}, \textbf{Bash}, \textbf{UML}, \textbf{Linux}, \textbf{I2C}, \textbf{Docker}, \textbf{Public Speaking}.
        }\\
    \twentyitem
        {Aug 2018 -}
    	{Jan 2018}
        {Embedded Software Engineer}
        {\href{https://www.montimage.com/}{Montimage, Paris, \textbf{France}}}
        {}
        {
            I led the integration of Montimage's MMT into IoT networks, demonstrating feasibility through a proof of concept and presenting impactful findings to a European consortium, resulting in the expansion of MMT's market reach with innovative features tailored for IoT security.
        }\\
    \twentyitem
        {Dec 2017 -}
  		{Sep 2017}
        {Teaching Assistant}
        {\href{http://www.scuolapsb.unina.it/}{University of Naples, \textbf{Italy}}}
        {}
        {
            At the University of Naples, I taught undergraduate students informatics and mathematics, crafting engaging lectures and exercises to facilitate their learning and build a solid academic foundation.
        }
\end{twenty}

\newpage
\makeextrainfo % Prints the sidebar

%----------------------------------------------------------------------------------------
%	 PROJECTS
%----------------------------------------------------------------------------------------
\section{Projects}

\textbf{Multi-Process Control System with Triple Modular Redundancy}
\vspace{1 mm}
\begin{itemize}
    \item Object oriented design and C coding of a framework for testing control laws of a spacecraft that carries cargo. Employed Linux inter-process primitives.
    \item Unit and integration tests executed in a containerised environment.
\end{itemize}

\textit{Skills}: Object oriented design, C, Docker.

\textbf{GNSS Synchronisation \& Bistatic Passive Radar}

\begin{itemize}
    \item Design and VHDL coding of the system in charge of finding the strongest doppler frequency among a given set of visible satellites in an embedded
    \item Object oriented design, embedded C development of Board Support Package and Linux device drivers.
    \item Unit and integration tests executed on Xilinx Zynq-7010 SoC.
\end{itemize}

\textit{Skills}: Object oriented design, C, VHDL, Linux, Device Drivers, Xilinx, GHDL.

\textbf{Arithmetic Logic Unit Implementation}

\begin{itemize}
    \item Design and VHDL coding of an integer Arithmetic Logic Unit (ALU) for Xilinx FPGAs. Project available on Github.
    \item Unit and integration tests executed via GHDL and then on Xilinx Spartan 3E FPGA.
\end{itemize}

\textbf{\textit{Skills}}: VHDL, GHDL, Xilinx Spartan 3E FPGA, Git, Github, Github Actions.

%----------------------------------------------------------------------------------------
%	 RESEARCH
%----------------------------------------------------------------------------------------
\section{Research}
\begin{twenty}
	\twentyitem
        {Aug 2021 -}
    	{Jun 2021}
        {Professional Degree in Space Studies (SSP21)}
        {\href{https://www.isunet.edu/ssp/}{ISU, Strasbourg, \textbf{France}}}
        {}
        {
            \vspace{1 mm}
            \textbf{Specialisation}: \textit{Management and Business}. In a team of 7, presented \textit{Astrolaunch}. Produced business plan, pitch deck and teaser. Presented the idea in front of jury of investors and experts.

            \vspace{1 mm}
            \textbf{Team Project}: \textit{Moon on-orbit Nexus providing orbital rendezvous and transportation}, MOONPORT. Innovative space transportation system to the cis-lunar space. Leading the Science team.
        }\\
	\twentyitem
        {Jan 2019 -}
    	{Sep 2018}
        {M.Sc. Candidate}
        {\href{http://www.scuolapsb.unina.it/}{University of Naples, Federico II, \textbf{Italy}}}
        {}
        {
           Proposed an Intrusion Detection System (IDS) for 6LoWPAN-based IoT networks in collaboration with the Montimage company and in the context of the ANASTACIA EU project.

           \vspace{1 mm}
           \textbf{Thesis}: \textit{Design of an Intrusion Detection System for IoT Security}
        }
\end{twenty}

%----------------------------------------------------------------------------------------
%	 PUBLICATIONS
%----------------------------------------------------------------------------------------
\section{Publications}
    \href{https://www.researchgate.net/publication/355856343_MOONPORT_A_cost-effective_transport_solution_for_cislunar_space}
    {\textbf{MOONPORT: a cost-effective transport solution for cislunar space}}.\\
    F. Bergamasco, \href{https://scholar.google.com/citations?user=A3XqqTEAAAAJ&hl=it}{A. Riccio}, et. al.,
    \href{https://iafastro.directory/iac/paper/id/63931/summary/}
    {72nd International Astronautical Congress}, 2021.

    \textbf{A security monitoring system for Internet of Things}.
    V. Casola,  \href{https://scholar.google.com/citations?user=A3XqqTEAAAAJ&hl=it}{A. Riccio}, et al.,\\
    \href{https://www.researchgate.net/publication/334175322_A_security_monitoring_system_for_Internet_of_Things}
    {Internet of Things: Engineering Cyber Physical Human Systems}, 2019.
\end{document}
