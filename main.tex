%!LW recipe=latexmk (xelatex)
%%%%%%%%%%%%%%%%%%%%%%%%%%%%%%%%%%%%%%%%%
% Twenty Seconds Resume/CV
% LaTeX Template
% Version 1.0 (14/7/16)
%
% Original author:
% Carmine Spagnuolo (cspagnuolo@unisa.it) with major modifications by
% Vel (vel@LaTeXTemplates.com) and Harsh (harsh.gadgil@gmail.com)
%
% License:
% The MIT License (see included LICENSE file)
%
%%%%%%%%%%%%%%%%%%%%%%%%%%%%%%%%%%%%%%%%%

%----------------------------------------------------------------------------------------
%	PACKAGES AND OTHER DOCUMENT CONFIGURATIONS
%----------------------------------------------------------------------------------------

\documentclass[letterpaper]{twentysecondcv} % a4paper for A4

%----------------------------------------------------------------------------------------
%	 PERSONAL INFORMATION
%----------------------------------------------------------------------------------------

\cvname{Antonio Riccio}
\cvjobtitle{\textbf{M.Sc. Eng. Embedded} \\ \textbf{Software Engineer}}

\profilepic{img/headshot.jpg}
\cvnationality{}
\cvaddress{}
\cvnumberphone{(+49) 151 56893596}
\cvmail{careers@ariccio.me}
\cvsite{bit.ly/48mf8xO}
\cvsiteshown{ariccio.me}
\cvlinkedin{antonioriccio}
\linkedinnameshown{Antonio Riccio}
\cvgithub{}
\cvtwitter{}
\cvresearchgate{}
\aboutme{\summarytext}

\newcommand{\summarytext}[1]
{
    Experienced Embedded Software Engineer specialising in software development for real-time, multithreading embedded systems in aerospace, defence, and telecommunications. Proficient in MISRA-compliant embedded C, C++ and VHDL, with hands-on expertise in Linux kernel development, device drivers, and FPGA firmware. Skilled in designing, implementing, and testing high-reliability software architectures for SoC, FPGA and microcontroller platforms. Fluent in leveraging Agile methodologies, UML for software modelling, and automation tools for CI/CD.
}

%----------------------------------------------------------------------------------------
%	INTERESTS
%----------------------------------------------------------------------------------------

\interests{
	\smartdiagram[bubble diagram]{
        \textbf{Embedded}\\\textbf{Systems},
        \textbf{~~~~~~Linux~~~~~~}\\\textbf{~~~~~Kernel~~~~~},
        \textbf{~~~~FPGA~~~~}\\\textbf{~~~SoC~~~},
        \textbf{~~~~~~Safety~~~~~~}\\\textbf{~~~Systems~~~},
        \textbf{~~~~RTOS~~~~},
        \textbf{~~~~Avionics~~~~},
        \textbf{Internet of}\\\textbf{~~Things~~}
    }
}

%----------------------------------------------------------------------------------------
%	SKILLS
%----------------------------------------------------------------------------------------

\keyskills{
    \begin{skilltable}
        \skillitem
        {\textbf{Programming Languages}: C, C++, VHDL, MIPS/M68k Assembler, Bash}
        \skillitem
        {\textbf{(MP)SoC/MCU}: Zynq UltraScale+ XCZU9EG MPSoC, Zynq-7010 SoC, STM32F401C-DISCO MCU, STM32F401RE MCU, i.MX 6Solo MCU, Raspberry Pi 4B MCU, Elegoo UNO R3}
        \skillitem
        {\textbf{FPGA/CPU}: Spartan-3E XC3S1200E FPGA, PolarFire FPGA, proASIC3e A3PE3000-FG484I FPGA, RTAX-S/SL RTAX2000S FPGA, ARM Cortex-A53/A72/A9, ARM Cortex-R5F, ARM Cortex-M4, ATmega328}
        \skillitem
        {\textbf{Protocols}: I2C, SPI, USB, U(S)ART, AMBA AHB/APB, AMBA AXI, IPv4/v6, TCP, UDP, CoAP, 6LoWPAN}
        \skillitem
        {\textbf{Operative Systems}: Linux, freeRTOS, Linux RT, RTEMS, Contiki, Windows}
        \skillitem
        {\textbf{Software Development Methodologies}: V Model, Agile, UML, Software/Hardware Design Patterns}
        \skillitem
        {\textbf{Software Tools}: Yocto, Petalinux, Git, SVN, Gerrit, Github Actions, STL, GDB, Docker, GHDL}
        \skillitem
        {\textbf{Software Development Standards}: MISRA-C:2004, RTCA-DO178, Linux Coding Style, ECSS-Q-ST-60-02C, ASIC/001}
    \end{skilltable}
}

%----------------------------------------------------------------------------------------
%	LANGUAGES
%----------------------------------------------------------------------------------------

\languages{
    \begin{center}
        \worldflag[stretch=20, length=0mm, width=5mm]{IT} (m)
        \worldflag[stretch=20, length=0mm, width=5mm]{GB} (C2)
        \worldflag[stretch=20, length=0mm, width=5mm]{DE} (B1)
    \end{center}
}

%----------------------------------------------------------------------------------------
%	CERTIFICATIONS
%----------------------------------------------------------------------------------------

\certifications{
    \href{https://www.cambridgeenglish.org/exams-and-tests/advanced/}{\textbf{Certificate in Advanced English (CAE)}} |
    \textit{Cambridge Assessment English}

    \href{https://www.isunet.edu/event/executive-space-course-2024/}{\textbf{Executive Space Course}} |
    \textit{ISU}

    \href{https://linkedin.com/learning}{\textbf{Advanced Topics with C++}} |
    \textit{Linkedin}

    \href{https://linkedin.com/learning}{\textbf{Software Design Patters}} |
    \textit{Linkedin}

    \href{https://udemy.com}{\textbf{VSD - Static Timing Analysis}} | \textit{Udemy}
}

%----------------------------------------------------------------------------------------
%	EDUCATION
%----------------------------------------------------------------------------------------

\education{
    \textbf{M.Sc. Computer Engineering} (\textit{laude})\\
    University Federico II, Italy | 2019

    \textbf{B.Sc. Computer Engineering} \\
    University Federico II, Italy | 2015
}

%----------------------------------------------------------------------------------------
%	VOLUNTEERING
%----------------------------------------------------------------------------------------

\volunteering{
    \href{https://spacebrewery.com/}{\textbf{SpaceBrewery}} |
    Organising networking events for space enthusiasts.

    \href{http://www.erasmusnapoli.it/}{\textbf{Erasmus Student Association (ESA)}} |
    Supporting incoming Erasmus students.

    \href{https://www.goodgym.org/about}{\textbf{GoodGym Slough}} |
    Charity running \\ group supporting local community.
}

%----------------------------------------------------------------------------------------
%	RESEARCH PAPERS
%----------------------------------------------------------------------------------------

\researchpapers{
    \href{https://www.researchgate.net/publication/355856343_MOONPORT_A_cost-effective_transport_solution_for_cislunar_space}
    {\textbf{MOONPORT: a cost-effective transport solution for cislunar space}}.
    \href{https://iafastro.directory/iac/paper/id/63931/summary/}
    {72nd International Astronautical Congress.}

    \textbf{A security monitoring system for Internet of Things.}
    \href{https://www.researchgate.net/publication/334175322_A_security_monitoring_system_for_Internet_of_Things}
    {Internet of Things: Engineering Cyber Physical Systems.}
}

%--------------------- DOCUMENT BEGINS --------------------------------------------------

\begin{document}
\makeprofile % Prints the sidebar

%
%----------------------------------------------------------------------------------------
%	 PROFESSIONAL EXPERIENCE
%----------------------------------------------------------------------------------------
%
\section{Professional Experience}

\begin{twenty} % Environment for a list with descriptions
    % \twentyitem
    %     {May 2022 -}
    %     {now}
    %     {TITLE}
    %     {\href{URL}{COMPANY, CITY, \textbf{COUNTRY}}}
    %     {}
    %     {
    %         BRIEF DESCRIPTION
    %         \vspace{1 mm}
    %         \begin{itemize}
    %             \item item 1
    %             \item item 2
    %             \item item 3
    %             \item item 4
    %             \item item 5
    %         \end{itemize}

    %         \vspace{1 mm}
    %         \textit{Keywords}: KEYWORDS
    %     }
    \twentyitem
        {Nov 2023 -}
        {May 2023}
        {Embedded Software Engineer}
        {\href{https://www.exploration.space/}{The Exploration Company, Munich, \textbf{Germany}}}
        {}
        {
            \vspace{2 mm}
            \underline{\textbf{Project}: spacecraft carrying cargo to Earth orbit.}

            \vspace{2 mm}
            \begin{itemize}
                \item UML architectural and detailed design,  MISRA-compliant embedded C development of software that collects data from different sensors and sends commands to actuators. Employed \textit{socat} library.
                \item Unit and integration in a Docker environment.  System tests on a Zynq UltraScale+ XCZU9EG MPSoC hardware platform. Employed Petalinux for Linux filesystem customisation and Github Actions for static analysis and cross-compilation.
            \end{itemize}

            \vspace{2 mm}
            \underline{\textbf{Project}: control system with Triple Modular Redundancy (TMR)}

            \vspace{2 mm}
            \begin{itemize}
                \item Object-oriented design and C coding of a framework for simulating flight control laws. Employed the Linux inter-process-comms APIs.
                \item Unit and integration tests executed in a Docker environment.
            \end{itemize}

            \vspace{2 mm}
            \textbf{\textit{Skills}}: MISRA, object-oriented design, C, Linux, socat, Docker, Petalinux, Agile, Github Actions.
        }\\
    \twentyitem
        {Apr 2023 -}
        {May 2022}
        {Embedded Software Engineer}
        {\href{https://www.airbus.com/en/who-we-are}{Airbus Defence \& Space, Manching, \textbf{Germany}}}
        {}
        {
            \vspace{2 mm}
            \underline{\textbf{Project}: embedded software for military airplane}

            \vspace{2 mm}
            \begin{itemize}
                \item ADA development for the DO-178-compliant DAL-A software controlling the flaps of the wings. Design and XML development of black/grey box test cases. Performed requirement reviews, regression tests execution on simulation and HW target. Employed the Modified Condition/Decision (MC/DC) coverage criteria.
                \item Manual inspection of intermediate object code, M68K and MIPS assembler to spot potential safety hazards.
            \end{itemize}

            \vspace{2 mm}
            \underline{\textbf{Project}: GNSS synchronisation \& bistatic passive radar}

            \vspace{2 mm}
            \begin{itemize}
                \item Design and VHDL coding of the system searching for the strongest Doppler frequency among a given set of visible satellites.
                \item Object-oriented design, embedded C development of Board Support Package (BSP) and Linux device drivers.
                \item Design and C coding of tests for the Xilinx Zynq-7010 SoC.
            \end{itemize}

            \vspace{2 mm}
            \textbf{\textit{Skills}}: Object-oriented design, C, ADA, VHDL, Linux, Device Drivers, GHDL, MIPS/M68k Assembler, DO-178, Unit Testing, Static Analysis.
        }\\
    \twentyitem
        {Oct 2021 -}
    	{May 2020}
        {Digital Design Engineer}
        {\href{http://evoleotech.com/company/}{EVOLEO Technologies, Munich, \textbf{Germany}}}
        {}
        {
            \vspace{2 mm}
            \underline{\textbf{Project}: FPGA image acquisition system}

            \vspace{2 mm}
            \begin{itemize}
                \item ECSS-Q-ST-60-02C compliant UML architectural and detailed design, VHDL development of an FPGA firmware for controlling a set of cameras for image acquisition. Integrated ECSS-E-ST-50-52C RMAP, SpaceWire, AMBA 3 AHB-Lite IP Cores.
                \item Design and VHDL development of unit tests, integration tests and system tests for Microsemi proASIC3e and RTAX-S/SL platforms.
            \end{itemize}

            \vspace{2 mm}
            \underline{\textbf{Project}: COTS on-board computer for satellites}

            \vspace{2 mm}
            \begin{itemize}
                \item UML architectural and detailed design, VHDL and C development of FPGA firmware and software for the platform controlling the entire satellite. Working with the CCSDS 133.0-B-2 protocol.
                \item Design and VHDL development of unit tests, integration tests and system tests for Zynq UltraScale+ XCZU9EG MPSoC and Microsemi PolarFire FPGA.
            \end{itemize}

            \vspace{2 mm}
            \textbf{\textit{Skills}}: C, VHDL, UML, MPSoC, FPGA, firmware, SpaceWire, ECSS.
        }
\end{twenty}

\newpage
\makeextrainfo % Prints the sidebar

%----------------------------------------------------------------------------------------
%	 PROFESSIONAL EXPERIENCE
%----------------------------------------------------------------------------------------
\section{Professional Experience}

\begin{twenty}
    \twentyitem
    	{Apr 2020 -}
		{Jan 2018}
        {Embedded Software Engineer}
        {\href{https://www.airspan.com/}{Airspan, London, \textbf{UK}}}
        {}
        {
            \vspace{2 mm}
            \underline{\textbf{Project}: LTE (Long-Term Evolution) WiFi backhaul system}

            \vspace{2 mm}
            \begin{itemize}
                \item UML architectural and detailed design, object-oriented C programming of user-space software on top of Linux OS. Employed UNIX Domain Sockets, \textit{tcpdump} and \textit{iperf} libraries.
                \item Design and C++ coding of unit tests, regressions tests and system tests on NXP .i.MX 6Solo microcontroller hardware platform.
            \end{itemize}

            \vspace{2 mm}
            \underline{\textbf{Project}: 5G virtualized RAN (Radio Access Network) system}

            \vspace{2 mm}
            \begin{itemize}
                \item Architectural and detailed design, bash development of a Docker-based emulation environment for a new hardware platform.
            \end{itemize}

            \vspace{2 mm}
            \underline{\textbf{Project}: Porting of the Linux kernel on a Xilinx Zynq hardware platform}

            \vspace{2 mm}
            \begin{itemize}
                \item Detailed design and C development of device drivers for the Linux kernel. Employed Yocto, I2C, SPI and the Input subsystem APIs.
                \item Design and C coding of unit tests, regressions tests and system tests on a Xilinx Zynq-7000 SoC hardware platform.
            \end{itemize}

            \vspace{2 mm}
            \textbf{\textit{Skills}}: C, C++, Bash, UML, Linux Kernel, Device Drivers, I2C, SPI, Input subsystem, Zynq-7000 SoC, i.MX 6Solo MCU, Docker, Yocto.
    }\\
    \twentyitem
        {Dec 2017 -}
        {Jan 2017}
        {Embedded Software Engineer Intern}
        {\href{https://www.montimage.com/}{Montimage, Paris, \textbf{France}}}
        {}
        {
            \vspace{2 mm}
            \underline{\textbf{Project}: Porting of an Intrusion Detection System (IDS) to IoT networks}

            \vspace{2 mm}
            \begin{itemize}
                \item Model-based UML detailed design, object-oriented C programming of user-space software on top of the 6LoWPAN/CoAP stack for the Linux OS. Employed the UNIX Socket library.
                \item Porting of the software for to the Contiki OS. Systems tests on the Cooja network simulator. Employed \textit{tcpdump} library.
            \end{itemize}

            \vspace{2 mm}
            \textbf{\textit{Skills}}: C, C++, 6LoWPAN, CoAP, UML, Linux, tcpdump, Git, Github.
        }
\end{twenty}

%----------------------------------------------------------------------------------------
%	 PERSONAL PROJECTS
%----------------------------------------------------------------------------------------
\section{Personal Projects}

\textbf{Arithmetic Logic Unit Implementation}

\begin{itemize}
    \item Design and VHDL coding of an integer Arithmetic Logic Unit (ALU) for Xilinx FPGAs. Project available on Github.
    \item Unit and integration tests executed via GHDL and then on Xilinx Spartan 3E FPGA.
\end{itemize}

\textbf{\textit{Skills}}: VHDL, GHDL, Xilinx Spartan 3E FPGA, Git, Github, Github Actions.

%----------------------------------------------------------------------------------------
%	 EDUCATION AND QUALIFICATIONS
%----------------------------------------------------------------------------------------
\section{Education and Qualifications}
\begin{twenty}
	\twentyitem
        {2021}
    	{}
        {Professional Degree in Space Studies (SSP21)}
        {\href{https://www.isunet.edu/ssp/}{ISU, Strasbourg, \textbf{France}}}
        {}
        {
            Three months summer course; interdisciplinary program designed to provide professionals with a broad understanding of space-related fields, including engineering, policy, science, and business.
        }\\
	\twentyitem
        {2017}
    	{}
        {M.Sc. Computer Engineering}
        {\href{http://www.scuolapsb.unina.it/}{University of Naples, Federico II, \textbf{Italy}}}
        {}
        {
            Dept. of Computer Engineering, specialization in Embedded Software, three years course study; final grade 110/110 \textit{cum laude} (i.e. 110 points out of a possible 110 with honors).
        }
\end{twenty}

%----------------------------------------------------------------------------------------
%	 LANGUAGES
%----------------------------------------------------------------------------------------
\section{Languages}

\begin{itemize}
    \item German, intermediate
    \item English, fluent (lived in the UK, Cambridge Advanced English Certificate (CAE))
    \item Italian, mothertongue
\end{itemize}

\end{document}
