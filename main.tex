%!LW recipe=latexmk (xelatex)
%%%%%%%%%%%%%%%%%%%%%%%%%%%%%%%%%%%%%%%%%
% Twenty Seconds Resume/CV
% LaTeX Template
% Version 1.0 (14/7/16)
%
% Original author:
% Carmine Spagnuolo (cspagnuolo@unisa.it) with major modifications by
% Vel (vel@LaTeXTemplates.com) and Harsh (harsh.gadgil@gmail.com)
%
% License:
% The MIT License (see included LICENSE file)
%
%%%%%%%%%%%%%%%%%%%%%%%%%%%%%%%%%%%%%%%%%

%----------------------------------------------------------------------------------------
%	PACKAGES AND OTHER DOCUMENT CONFIGURATIONS
%----------------------------------------------------------------------------------------

\documentclass[letterpaper]{twentysecondcv} % a4paper for A4

%----------------------------------------------------------------------------------------
%	 PERSONAL INFORMATION
%----------------------------------------------------------------------------------------

\cvname{Antonio Riccio}
\cvjobtitle{\textbf{Embedded Systems} \\ \textbf{Engineer}}

\profilepic{img/headshot.jpg}
\cvnationality{}
\cvaddress{}
\cvnumberphone{(+49) 151 56893596}
\cvmail{careers@ariccio.me}
\cvsite{bit.ly/48mf8xO}
\cvsiteshown{ariccio.me}
\cvlinkedin{antonioriccio}
\linkedinnameshown{Antonio Riccio}
\cvgithub{}
\cvtwitter{}
\cvresearchgate{}
\aboutme{\summarytext}

\newcommand{\summarytext}[1]
{
    Ich kombiniere meine Programmierkenntnisse mit einem nomadischen Geist. Als Softwareingenieur erkunde ich verschiedene Bereiche, von Finanzen bis hin zu Geschäftsführung, mit dem Ziel, Führungsrollen zu übernehmen und \\meine Leidenschaft für neue Herausforderungen zu nutzen.
}

%----------------------------------------------------------------------------------------
%	INTERESTS
%----------------------------------------------------------------------------------------

\interests{
	\smartdiagram[bubble diagram]{
        \textbf{Embedded}\\\textbf{Systems},
        \textbf{~~~~~~Linux~~~~~~}\\\textbf{~~~~~Kernel~~~~~},
        \textbf{~~~~FPGA~~~~}\\\textbf{~~~SoC~~~},
        \textbf{~Sicherheits~}\\\textbf{~systeme~},
        \textbf{~~~~RTOS~~~~},
        \textbf{~~~~Avionik~~~~},
        \textbf{Internet der}\\\textbf{~~Dinge~~}
    }
}

%----------------------------------------------------------------------------------------
%	SKILLS
%----------------------------------------------------------------------------------------

% each skill must have a value between 0 an 6 (float)
\skills{
    {C $\textbullet$ Bash $\textbullet$ VHDL $\textbullet$ Linux $\textbullet$ C++/4.5},
    {Java $\textbullet$ MCU $\textbullet$ SoC $\textbullet$ FPGA/4},
    {I2C $\textbullet$ SPI $\textbullet$ USART $\textbullet$ AXI $\textbullet$ USB/3.8},
    {Markdown $\textbullet$ Doxygen $\textbullet$ \LaTeX/3.5},
    {CPU $\textbullet$ Assembler $\textbullet$ RISC/3.3},
    {SystemVerilog $\textbullet$ freeRTOS $\textbullet$ RTAI/3}}

\softskills{
    \smartdiagram[bubble diagram]{
        \textbf{Teamarbeit},
        \textbf{Kommunikation},
        \textbf{Problem-}\\\textbf{~lösung~},
        \textbf{~Kritisches~}\\\textbf{~Denken~},
        \textbf{~~Führung~~},
        \textbf{Zeit-}\\\textbf{~management~},
        \textbf{Lern-}\\\textbf{~fähigkeit~}
    }
}

%----------------------------------------------------------------------------------------
%	LANGUAGES
%----------------------------------------------------------------------------------------

\languages{
    \begin{center}
        \worldflag[stretch=20, length=0mm, width=5mm]{IT} (m)
        \worldflag[stretch=20, length=0mm, width=5mm]{GB} (C2)
        \worldflag[stretch=20, length=0mm, width=5mm]{DE} (B1)
    \end{center}
}

%----------------------------------------------------------------------------------------
%	CERTIFICATIONS
%----------------------------------------------------------------------------------------

\certifications{
    \href{https://www.cambridgeenglish.org/exams-and-tests/advanced/}{\textbf{Certificate in Advanced English (CAE)}} |
    \textit{Cambridge Assessment English}

    \href{https://www.isunet.edu/event/executive-space-course-2024/}{\textbf{Executive Space Course}} |
    \textit{ISU}

    \href{https://coursera.org/share/021a99618cdf1997cabe7ab5b354c066}{\textbf{Finanzmärkte}} |
    \textit{Coursera}

    \href{https://courses.edx.org/certificates/1952c815bf1a48b09b4bdc7e5ec62587}{\textbf{Space Missions Design \& Ops}} | \textit{edX}

    \href{https://courses.edx.org/certificates/fe4896ee42de4e28a9951475845face8}{\textbf{Aerospace Eng.: Astronautics}} | \textit{edX}
}

%----------------------------------------------------------------------------------------
%	EDUCATION
%----------------------------------------------------------------------------------------

\education{
    \textbf{M.Sc. Computer Engineering} (\textit{laude})\\
    Universität Federico II, Italien | 2019

    \textbf{B.Sc. Computer Engineering} \\
    Universität Federico II, Italien | 2015
}

%----------------------------------------------------------------------------------------
%	VOLUNTEERING
%----------------------------------------------------------------------------------------

\volunteering{
    \href{https://spacebrewery.com/}{\textbf{SpaceBrewery}} |
    Networking-Events für Raumfahrt-Enthusiasten.

    \href{http://www.erasmusnapoli.it/}{\textbf{Erasmus Student Association (ESA)}} |
    Unterstützung von Erasmus-Studenten.

    \href{https://www.goodgym.org/about}{\textbf{GoodGym Slough}} |
    Unterstützung der örtlichen Gemeinde.
}

%--------------------- DOCUMENT BEGINS --------------------------------------------------

\begin{document}
\makeprofile % Prints the sidebar

%
%----------------------------------------------------------------------------------------
%	 EXPERIENCE
%----------------------------------------------------------------------------------------
%
\section{Berufserfahrung}

\begin{twenty} % Environment for a list with descriptions
    % \twentyitem
    %     {May 2022 -}
    %     {now}
    %     {TITLE}
    %     {\href{URL}{COMPANY, CITY, \textbf{COUNTRY}}}
    %     {}
    %     {
    %         BRIEF DESCRIPTION
    %         \vspace{1 mm}
    %         \begin{itemize}
    %             \item item 1
    %             \item item 2
    %             \item item 3
    %             \item item 4
    %             \item item 5
    %         \end{itemize}

    %         \vspace{1 mm}
    %         \textit{Keywords}: KEYWORDS
    %     }
    \twentyitem
        {Nov 2023 -}
        {May 2023}
        {Embedded Software Engineer}
        {\href{https://www.exploration.space/}{The Exploration Company, München, \textbf{De}}}
        {}
        {
            Entwicklung von Flugsteuerungssoftware für das Raumfahrzeug \textit{Mission Possible}.
            \vspace{1 mm}
            \begin{itemize}
                \item Entwicklung von Anwendungen unter Verwendung des cFS-Frameworks der NASA.
                \item Sicherstellung der Code-Konformität gemäß den MISRA- und JPL-Richtlinien.
                \item Erfolgreiches Testen der Anwendungen mithilfe von seriellen Emulationstools und Python-Skripten, um die Funktionalität der Software auch ohne physische Sensoren nachzuweisen.
            \end{itemize}

            \vspace{1 mm}
            \textit{Keywords}: \textbf{C}, \textbf{Python}, \textbf{Linux}, \textbf{Docker}, \textbf{Hardware-in-the-loop}, \textbf{Agile}.
        }\\
    \twentyitem
        {Apr 2023 -}
        {May 2022}
        {Embedded Software Engineer}
        {\href{https://www.airbus.com/en/who-we-are}{Airbus Defence \& Space, Manching, \textbf{De}}}
        {}
        {
            Entwicklung von Flugsteuerungssoftware für das Kampfflugzeug \textbf{Eurofighter Typhoon}.
            \vspace{1 mm}
            \begin{itemize}
                \item Entwicklung und Aktualisierung von Test-Suites, um 100\% Abdeckung und MC/DC-Kriterien zu gewährleisten.
                \item Durchführung manueller statischer Analysen des Quellcodes und Assembler zur Identifizierung potenzieller Sicherheitsrisiken.
                \item Erfolgreiche Integration einer kritischen Komponente, die die Leistung und Funktionalität des Flugzeugs verbessert hat.
            \end{itemize}

            \vspace{1 mm}
            \textit{Keywords}: \textbf{ADA}, \textbf{Assembler}, \textbf{DO-178}, \textbf{Unit Testing}, \textbf{Static Analysis}.
        }\\
    \twentyitem
        {Oct 2021 -}
    	{May 2020}
        {Digital Design Engineer}
        {\href{http://evoleotech.com/company/}{EVOLEO Technologies, Munich, \textbf{De}}}
        {}
        {
            Führung bei FPGA- und SoC-Lösungen für Satellitenanwendungen.
            \vspace{1 mm}
            \begin{itemize}
                \item Leitung der digitalen Designarbeiten für die \href{https://sci.esa.int/web/plato}{ESA-PLATO-Mission}, Optimierung der Entwicklungsprozesse und Implementierung strenger Testmethoden zur Einhaltung der ESA-Standards.
                \item Erfolgreiches Bestehen sowohl des Preliminary Design Reviews als auch des Critical Design Reviews, mit Beantwortung von Bedenken und Sicherstellung der Projekterfüllung.
                \item Übernahme einer Führungsrolle in einem bahnbrechenden Projekt zur Entwicklung eines Satellitenbordcomputers unter Verwendung handelsüblicher Komponenten, Koordinierung digitaler Designaktivitäten und Zusammenarbeit mit Softwareteams zur nahtlosen Integration.
            \end{itemize}

            \vspace{1 mm}
            \textit{Keywords}: \textbf{VHDL}, \textbf{C}, \textbf{SpaceWire}, \textbf{RMAP}, \textbf{CCSDS}, \textbf{ECSS}, \textbf{DSP}, \textbf{COTS}, \textbf{Xilinx US+}, \textbf{Leadership}, \textbf{Project Management}, \textbf{Mentorship}.
        }\\
    \twentyitem
    	{Apr 2020 -}
		{May 2019}
        {Embedded Software Engineer}
        {\href{https://www.airspan.com/}{Airspan, London, \textbf{UK}}}
        {}
        {
            Optimierung von Testprozessen und Verbesserung der Software-Skalierbarkeit.
            \vspace{1 mm}
            \begin{itemize}
                \item Entwicklung einer Lösung zur Automatisierung von GPS-Modultests, um deren nahtlose Integration in das Testframework des Unternehmens zu ermöglichen.
                \item Entwicklung einer Lösung zur Verwaltung verschiedener Touchscreens über die I2C-Schnittstelle und Linux-Systemaufrufe unter Verwendung von objektorientiertem C-Programmieren zur Verbesserung der Softwareflexibilität und -skalierbarkeit.
                \item Leitung der Erstellung einer virtualisierten Umgebung zur Simulation von Hardwareplattformen, um eine ununterbrochene Anwendungstests ohne physische Hardware zu ermöglichen.
            \end{itemize}

            \vspace{1 mm}
            \textit{Keywords}: \textbf{C}, \textbf{C++}, \textbf{Bash}, \textbf{UML}, \textbf{Linux}, \textbf{I2C}, \textbf{Docker}, \textbf{Public Speaking}.
        }\\
    \twentyitem
        {Aug 2018 -}
    	{Jan 2018}
        {Embedded Software Engineer}
        {\href{https://www.montimage.com/}{Montimage, Paris, \textbf{Frankreich}}}
        {}
        {
            Leitung der Integration von Montimages MMT in IoT-Netzwerke und Demonstration der Machbarkeit durch ein Proof-of-Concept. Präsentation der Ergebnisse vor einem europäischen Konsortium, was zur Erweiterung der Marktpräsenz von MMT mit innovativen IoT-Sicherheitsfunktionen führte.


        }
\end{twenty}

\newpage
\makeextrainfo % Prints the sidebar

%----------------------------------------------------------------------------------------
%	 PROJECTS
%----------------------------------------------------------------------------------------
\section{Projekte}

\textbf{Multi-Prozess-Steuerungssystem mit Triple Modular Redundancy (C)}

In meiner Freizeit entwickelte ich ein vielseitiges Steuerungssystem mit Sensoren, Aktuatoren und Controllern, die in einer Open-Loop-Konfiguration eingerichtet sind. Durch den Einsatz eines flag-basierten Mechanismus kann das System nahtlos zu einem Triple Modular Redundancy (TMR)-Schema wechseln und bietet durch Sensorreplikation und Datenfilterung erhöhte Zuverlässigkeit. Jede Systemkomponente ist einem unabhängigen Linux-Prozess zugeordnet, was die Modularisierung und Kommunikation über abstrahierte Kanäle, derzeit implementiert über \\ Linux-Nachrichtenwarteschlangen, erleichtert. Das System bietet Flexibilität in den Regelgesetzen, sodass diese ohne vollständige Neukompilierung des Codes ausgetauscht werden können.

\textbf{GNSS-Synchronisation \& Bistatisches Passivradar (VHDL, C)}

Ich initiierte eine Zusammenarbeit mit einem Verteidigungsunternehmen, um zur Entwicklung eines bistatischen Passivradars beizutragen. Dabei stellte ich mich der Herausforderung, die stärkste Doppler-Frequenz unter Satellitensignalen zu identifizieren. Durch innovative Techniken teilte ich die Aufgabe in zwei Unterblöcke auf, optimierte die Ressourcennutzung durch die Berechnung von Signalstärken „on-the-fly“ und integrierte den Xilinx DDS-Compiler zur flexiblen Signalgenerierung. Ich entwickelte außerdem Linux-Treiber für ein Software-Proof-of-Concept und führte Stakeholder erfolgreich in die Funktionalität ein.

\textbf{Arithmetic Logic Unit Implementierung (VHDL)}

Als Leiter und integrales Mitglied eines Universitätsprojekts leitete ich die Entwicklung und Implementierung einer Arithmetic Logic Unit (ALU) auf einem Xilinx Spartan 3E FPGA. Meine Aufgaben reichten von der Projektüberwachung bis hin zur Entwicklung kritischer Komponenten. Durch sorgfältige Evaluierung und Auswahl optimaler Lösungen zeigte ich Fachkenntnisse in der Hardware-Entwicklung und im Projektmanagement.



%----------------------------------------------------------------------------------------
%	 RESEARCH
%----------------------------------------------------------------------------------------
\section{Forschung}
\begin{twenty}
	\twentyitem
        {Aug 2021 -}
    	{Jun 2021}
        {Professioneller Abschluss in Weltraumstudien}
        {\href{https://www.isunet.edu/ssp/}{ISU, Straßburg, \textbf{Frankr.}}}
        {}
        {
            \vspace{1 mm}
            \textbf{Spezialisierung}: \textit{Management and Business}. In einem Team von 7 Personen habe ich das Projekt \textbf{Astrolaunch} präsentiert.

            \vspace{1 mm}
            \textbf{Teamprojekt}: \textit{Moon on-orbit Nexus providing orbital rendezvous and transportation}, MOONPORT. Innovatives Weltraumtransportsystem für den cislunaren Raum. Leitung des Wissenschaftsteams.
        }\\
	\twentyitem
        {Jan 2019 -}
    	{Sep 2018}
        {M.Sc. Kandidat}
        {\href{http://www.scuolapsb.unina.it/}{Universität Neapel, Federico II, \textbf{Italien}}}
        {}
        {
            Vorgeschlagenes Intrusion Detection System (IDS) für 6LoWPAN-basierte IoT-Netzwerke in Zusammenarbeit mit dem Unternehmen Montimage und im Kontext des EU-Projekts ANASTACIA.

           \vspace{1 mm}
           \textbf{Abschlussarbeit}: \textit{Design eines Intrusion Detection Systems für die IoT-Sicherheit.}
        }
\end{twenty}

%----------------------------------------------------------------------------------------
%	 PUBLICATIONS
%----------------------------------------------------------------------------------------
\section{Veröffentlichungen}
    \href{https://www.researchgate.net/publication/355856343_MOONPORT_A_cost-effective_transport_solution_for_cislunar_space}
    {\textbf{MOONPORT: a cost-effective transport solution for cislunar space}}.\\
    F. Bergamasco, \href{https://scholar.google.com/citations?user=A3XqqTEAAAAJ&hl=it}{A. Riccio}, et. al.,
    \href{https://iafastro.directory/iac/paper/id/63931/summary/}
    {72nd International Astronautical Congress}, 2021.

    \textbf{A security monitoring system for Internet of Things}.
    V. Casola,  \href{https://scholar.google.com/citations?user=A3XqqTEAAAAJ&hl=it}{A. Riccio}, et al.,\\
    \href{https://www.researchgate.net/publication/334175322_A_security_monitoring_system_for_Internet_of_Things}
    {Internet of Things: Engineering Cyber Physical Human Systems}, 2019.
\end{document}
